  \sectioncounter{1}
  \section{四种命题与充要条件}

  \subsection{知识梳理}
  可以判断真假的陈述句叫做\myindex{命题} (proposition), 
  其中判断为真的语句叫做\myemph{真命题}, 判断为假的语句叫做\myemph{假命题}.
  在判断命题的真假 (true or false) 时, 对真命题需要证明, 对假命题则只需要举出一个反例. 
  例如由不等式的性质知 ``若 $x>2$, $y>1$, 则 $x+y>3$'' 是真命题, 
  但 ``若 $x>2$, $y>1$, 则 $x-y>1$'' 是假命题, 
  因为可取 $x=3$, $y=5$ 满足前提但不满足结论. 后一个假命题也表明, \myemph{同向不等式不能相减}.
  
  形如 ``若 $p$, 则 $q$'' 的命题可以写出其逆命题、否命题和逆否命题. 四种命题的形式如下:
  \mymarginpar{逆命题和否命题也是互为逆否命题.}
  \begin{center}
    原命题: 若 $p$, 则 $q$;\quad 逆命题: 若 $q$, 则 $p$;\\
    否命题: 若 $\neg p$, 则 $\neg q$;\quad 逆否命题: 若 $\neg q$, 则 $\neg p$.
  \end{center}
  写否命题时, 一般只需要把原命题的判断词改为其否定形式, 比如 ``$=$'' 改为 ``$\neq$'', ``$>$'' 改为 ``$\leqslant$''.   注意, \myemph{原命题和否命题的真假没有必然联系}.
  \mymarginpar{同样也有例子说明原命题和逆命题的真假没有必然联系, 但是\myemph{原命题和其逆否命题却是同真假的}.}
  例如, ``若 $x=1$, 则 $x^2=1$'' 的否命题为 ``若 $x\neq 1$, 则 $x^2\neq 1$'', 前者真后者假 (为什么?); 而 ``若 $x=1$, 则 $x^3=1$'' 的否命题为 ``若 $x\neq 1$, 则 $x^3\neq 1$'', 两者都是真命题. 

  
  若条件 $p$ 能推出条件 $q$, 则称 $p$ 为 $q$ 的\myindex{充分条件} 
  (sufficient condition), $q$ 为 $p$ 的\myindex{必要条件} (necessary condition). 
  \mymarginpar{可简记为 ``充分'' 即是 ``已知'', ``必要'' 即是 ``结论''.}
  若条件 $p$, $q$ 能互推, 则称 $p$ 为 $q$ 的\myindex{充要条件} (necessary and sufficient condition), 也称 $p$, $q$ 等价, 此时 $q$ 也为 $p$ 的充要条件.
  例如, 设 $p\colon x>2$, $q\colon x>1$, 则 $p\Rightarrow q$ 但 $q\nRightarrow p$, 所以 $p$ 是 $q$ 的充分不必要条件, 而 $q$ 是 $p$ 的必要不充分条件. 
  
  判断条件 $p$ 对 $q$ 的充分性或必要性时, 可以先确定与两个条件描述对象等价的集合, 
  再由集合的包含关系判断充分性或必要性. 上例中 $p$ 对应区间 $A=(2,+\infty)$, 
  $q$ 对应区间 $B=(1,+\infty)$, 而 $A\subseteq B$ (参考数轴表示),
  所以 $p$ 是 $q$ 的充分不必要条件. 
  \mymarginpar{或简记为 (说法不严谨):\\
    \myemph{小集合是充分的, 大集合是必要的}.}
  判断多个条件之间的充分性或必要性时, 可以作出推理关系图, 见下面的练习~3.
    
  \lianxi
  \begin{exercise}
    将命题 ``斜率相等的两直线平行'' 改为 ``若 $p$, 则 $q$'' 的形式为\,? 
    它的逆否命题是\,?
  \end{exercise}
  
  \beginsolution
    原命题: 若两直线斜率相等, 则它们平行. 逆否命题: 若两直线平行, 则它们斜率相等.
  \endsolution
  
  \begin{exercise}
    判断下列命题的真假.
    
    (1) 命题 ``在 $\triangle ABC$ 中, 若 $AB>AC$, 则 $\angle C>\angle B$'' 和其否命题.
    
    (2) 命题 ``若 $ab=0$, 则 $b=0$'' 和其逆否命题.
  \end{exercise}

  \beginsolution
    (1) 否命题: 在 $\triangle ABC$ 中, 若 $\angle C>\angle B$, 则 $AB>AC$.
    
    由 ``大边对大角, 大角对大边'' 知, 原命题和其否命题均为真命题.
    
    (2) 逆否命题: 若 $b\neq 0$, 则 $ab\neq 0$.
    
    举反例可知, 原命题和其逆否命题均为假命题.
  \endsolution
  
  \begin{exercise}
    已知 $p$, $q$ 都是 $r$ 的必要条件, $s$ 是 $r$ 的充分条件, 
    $q$ 是 $s$ 的充分条件, 
    那么 $r$ 是 $q$ 的 $\underline{\qquad\qquad}$ 条件, 
    $p$ 是 $q$ 的 $\underline{\qquad\qquad}$ 条件.
  \end{exercise}

  \beginsolution
    作出推理关系图, 
    \mymarginpar{
    \begin{center}
    \begin{tikzpicture}[scale=1]
      \draw (0,1) node (s) {$s$};
      \draw (1,1) node (r) {$r$};
      \draw (0,0) node (q) {$q$};
      \draw (2,1) node (p) {$p$};
      \graph {(r)->(q)->(s)->(r)->(p)};
    \end{tikzpicture}
    \end{center}}
  可知 $r$ 是 $q$ 的充要条件, $p$ 是 $q$ 的必要条件.
  \endsolution
  
  \begin{exercise}
    已知 $p(x)\colon x^2 +2x-m>0$, 若 $p(1)$ 是假命题, $p(2)$ 是真命题,
    则实数 $m$ 的取值范围是\,?
  \end{exercise}

  \beginsolution
    由题: $1+2-m\leqslant 0$ 且 $4+4-m>0$, 解得 $m\in[3,8)$.
  \endsolution

  \subsection{要点导学\quad 各个击破}
  \subsubsection{命题的真假判定}
  \begin{example}
    设原命题为 ``若 $z_1$, $z_2$ 互为共轭复数, 则 $|z_1|=|z_2|$'',
    关于它的逆命题、否命题、逆否命题真假性的判断依次如下, 其中正确的是\,?(填序号)
    
    (1) 真、假、真;\qquad (2) 假、假、真;
    
    (3) 真、真、假;\qquad (4) 假、假、假.
  \end{example}

  \beginsolution
    逆命题: 若 $|z_1|=|z_2|$, 则 $z_1=\overline{z_2}$; 
    否命题: 若 $z_1\neq\overline{z_2}$, 则 $|z_1|\neq|z_2|$; 
    逆否命题: 若 $|z_1|\neq|z_2|$, 则 $z_1\neq\overline{z_2}$.
    
    因为 $z_1=\overline{z_2}\Leftrightarrow z_1$, $z_2$ 关于 $x$~轴对称, $|z_1|=|z_2|\Leftrightarrow z_1$, $z_2$ 到原点的距离相等, 所以逆命题和否命题均为假, 逆否命题为真.
  \endsolution
  
  \lianxi
  \begin{exercise}[s]
    下列有关命题的说法中正确的是\,?(填序号)
    
    (1) 命题 ``若 $x^2 =1$, 则 $x=1$'' 的否命题为 ``若 $x^2 =1$, 则 $x\neq 1$'';
    
    (2) ``$x=-1$'' 是 ``$x^2-5x-6=0$'' 的必要不充分条件;
    
    (3) 函数 $y=\sin\Bigl(2x+\dfrac{\pi}3\Bigr)$ 的最小正周期是 $\pi$;
    
    (4) ``$(x-3)(x-4)=0$'' 是 ``$x-3=0$'' 的充分不必要条件.
  \end{exercise}

  \beginsolution
    (1) 错误, 否命题为 ``若 $x^2\neq 1$, 则 $x\neq 1$''.
    
    (2) 错误, $x^2-5x-6=0\Leftrightarrow x=-1$ 或 $6$, 故 ``$x=-1$'' 是 ``$x^2-5x-6=0$'' 的充分不必要条件.
    
    (3) 正确, $T=\frac{2\pi}2= \pi$.
    
    (4) 错误, $(x-3)(x-4)=0\Leftrightarrow x-3=0$ 或 $x-4=0$, 故 ``$(x-3)(x-4)=0$'' 是 ``$x-3=0$'' 的必要不充分条件.
  \endsolution
  
  \subsubsection{充要条件的判定}
  \begin{example}
    设 $a$, $b\in \mathbb{R}$, 则 ``$a>b$'' 是 ``$a|a|>b|b|$'' 
    的$\underline{\qquad\qquad}$条件.
  \end{example}

  \beginsolution
    方法一: 分 $a>b\geqslant 0$, $a\geqslant 0>b$, $0>a>b$ 讨论.
    
    方法二: 作 $f(x)= x|x|= \begin{cases}
      x^2,& x\geqslant 0,\\
      -x^2,& x<0 \end{cases}$ 的图象知, $f(x)$ 为增函数.
    \mymarginpar{两种方法均可得, $a>b\Leftrightarrow a|a|>b|b|$.}
    
    \varexercise $a>b\Leftrightarrow a|a^3|>b|b^3|$.
    
    \varexercise 一般化, $a>b\Leftrightarrow a|a^{2n+1}|>b|b^{2n+1}|$, 其中 $n$ 为正整数.
  \endsolution
  
  \begin{example}
    已知 $p\colon x+2\geqslant 0$ 且 $x-10\leqslant 0$, 
    $q\colon 1-m\leqslant x\leqslant 1+m$, $m>0$.
    
    (1) 若 $m=1$, 则 $p$ 是 $q$ 的什么条件\,?
    
    (2) 若 $p$ 是 $q$ 的必要不充分条件, 求实数 $m$ 的取值范围.
  \end{example}

  \beginsolution
    $p\colon -2\leqslant x\leqslant 10$.
    
    (1) 若 $m=1$, 则 $q\colon 0\leqslant x\leqslant 2$, $p$ 是 $q$ 的必要不充分条件.
    
    (2) 由题: $1-m\leqslant -2< 10\leqslant 1+m$, 且等号不能同时成立, 则 $m\in[9,+\infty)$.
  \endsolution
  
  \subsubsection{课堂评价}
  \begin{exercise}
    ``$1<x<2$'' 是 ``$x<2$'' 成立的$\underline{\qquad\qquad}$条件.
  \end{exercise}

  \beginsolution
    ``必要不充分''.
  \endsolution
  
  \begin{exercise}
    已知条件 $p\colon 2^x \geqslant \Bigl(\dfrac12\Bigr)^x$, 
    $q\colon x^2 \geqslant -x$, 则 $p$ 是 $q$ 的$\underline{\qquad\qquad}$条件.
  \end{exercise}

  \beginsolution
    $p\colon x\geqslant 0$, $q\colon x\leqslant -1$ 或 $x\geqslant 0$, 则 $p$ 是 $q$ 的充分不必要条件.
  \endsolution
  
  \begin{exercise}
    ``$x<0$'' 是 ``$\ln(x+1)<0$'' 的$\underline{\qquad\qquad}$条件.
  \end{exercise}

  \beginsolution
    $\ln(x+1)<0\Leftrightarrow -1<x<0$, 填 ``必要不充分''. 
    \mymarginpar{注意对数不等式的解法.}
  \endsolution
  
  \begin{exercise}
    已知命题 ``关于 $x$ 的方程 $x^2 +x+a=0$ 有实根'' 为真命题, 求 $a$ 的取值范围.
  \end{exercise}

  \beginsolution
    判别式 $\Delta= 1-4a\geqslant 0$, 则 $a\in\Bigl(-\infty,-\frac14\Bigr]$.
  \endsolution
  
  \subsection{课后练习}
  \begin{exercise}
    命题 ``若函数 $f(x)=\log_ a x$ ($a>0$, $a\neq 1$) 是减函数,
    则 $\log_a 2<0$'' 的真假性是\,?
  \end{exercise}

  \beginsolution
    $f(x)\searrow$ 表明 $f(2)<f(1)$, 即 $\log_a 2<0$. 另: 也可先求得 $0<a<1$.
  \endsolution
  
  \begin{exercise}
    ``$\log_3 M>\log_3 N$'' 是 ``$M>N$'' 的$\underline{\qquad\qquad}$条件.
  \end{exercise}

  \beginsolution
    前者等价于 $M>N>0$, 为后者的充分不必要条件.
    \mymarginpar{对数式应注意真数一定大于 $0$.}
  \endsolution
  
  \begin{exercise}
    ``$m=1$'' 是 ``两直线 $x-y=0$ 和 $x+my=0$ 互相垂直'' 
    的$\underline{\qquad\qquad}$条件.
  \end{exercise}

  \beginsolution
    后者等价于 $1\cdot 1+(-1)\cdot m=0$, 即 $m=1$.
  \endsolution
  
  \begin{exercise}
    在 $\triangle ABC$ 中, ``$A=60^\circ$'' 是 ``$\cos A= \dfrac12$'' 
    的$\underline{\qquad\qquad}$条件.
  \end{exercise}

  \beginsolution
    $A\in(0^\circ, 180^\circ)$, 故 $A=60^\circ\Leftrightarrow\cos A= \dfrac12$.
    
    \varexercise 题中, 只把 ``$\cos A= \dfrac12$'' 换成 ``$\sin A= \dfrac{\sqrt3}2$''. 
    
    此时 $\sin A= \dfrac{\sqrt3}2\Leftrightarrow A=60^\circ$ 或 $120^\circ$, 答案改为 ``充分不必要''.
  \endsolution
  
  \begin{exercise}
    已知命题 $p$: 正数 $a$ 的平方不等于 $0$, 
    命题 $q$: 若 $a$ 不是正数, 则它的平方等于 $0$. 
    那么命题 $p$ 是命题 $q$ 的$\underline{\qquad\qquad}$.
    (填 ``逆命题''``否命题''``逆否命题'' 或 ``否定''.)
  \end{exercise}

  \beginsolution
    $p$: 若 $a$ 是正数, 则它的平方不等于 $0$. 填 ``否命题''.
  \endsolution
  
  \begin{exercise}
    ``$|x|\geqslant 1$'' 是 ``$x>2$'' 的$\underline{\qquad\qquad}$条件.
  \end{exercise}

  \beginsolution
    $|x|\geqslant 1\Leftrightarrow x\leqslant -1$ 或 $x\geqslant 1$, 填 ``不要不充分''.
  \endsolution
  
  \begin{exercise}
    判断命题 ``若 $a>b$, $ab\neq 0$, 则 $\frac1a<\frac1b$'' 的真假.
  \end{exercise}
  
  \beginsolution
    举反例 (如 $a=1$, $b=-1$) 可知该命题为假命题. 也可利用函数 $f(x)=\frac1x$ 分段递减的性质.
  \endsolution
  
  
  \begin{exercise}
    直线 $l\colon y=kx+1$ 与圆 $O\colon x^2 +y^2 =1$ 相交于 $A$, $B$ 两点, 
    则 ``$k=1$'' 是 ``$\triangle OAB$ 的面积为 $\dfrac12$'' 
    的$\underline{\qquad\qquad}$条件.
  \end{exercise}

  \beginsolution
    $l$ 过定点 $(0,1)$, 设为 $A$, 
    \mymarginpar{高中求三角形的面积时, 面积定理也很常用.}
    则 
    \begin{align*}
      S_{\triangle OAB}=\frac12 
      &\Leftrightarrow \frac12 |OA|\cdot|OB|\sin\angle AOB= \frac12 
       \Leftrightarrow \angle AOB= 90^\circ \\
      &\Leftrightarrow B(1,0) \text{\ 或\ } (-1,0) 
       \Leftrightarrow k=\pm1,
    \end{align*}
    填 ``充分不必要''.
  \endsolution
  
%%%%%%%%%%%%%%%%%%%%%%%%%%%